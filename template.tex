%!TEX TS-program = xelatex

\documentclass[a4paper,$fontsize$]{article}

% Adjust paper margins.
\usepackage{geometry}
\geometry{$geometry$}

% Disable page numbering.
\pagenumbering{gobble}

% Clean up footnote formatting.
\usepackage[hang,flushmargin,norule,symbol]{footmisc} % Must be loaded before ragged2e.

% Typeset entire document with left alignment.
\usepackage[document]{ragged2e}

% Load OpenType fonts in LaTeX document.
\usepackage{fontspec}

% Configure main and sans fonts.
$if(font)$
  \setmainfont{$font$}
$endif$

\usepackage{numprint} % Number formatting.
\usepackage{spreadtab} % Excel-like tables.
\usepackage{xpatch} % Needed for patching later.

% Configure hyperlinks and PDF metadata.
\usepackage{hyperref}
\hypersetup {
  hidelinks,
  pdfpagemode=FullScreen,
  pdfauthor={$issuer.name$},
  pdfsubject=Invoice \#$invoice-number$,
  pdftitle=Invoice \#$invoice-number$,
}

\usepackage{xcolor}
\usepackage{tabularx}

\definecolor{DimGray}{RGB}{105,105,105}
\definecolor{LightGray}{RGB}{211,211,211}

% Breathing room between table rows.
\renewcommand{\arraystretch}{1.15}

% Use thousands separator in spreadtab.
% https://groups.google.com/forum/#!topic/comp.text.tex/G8-ckv-MgZQ
\npthousandsep{,}
\npdecimalsign{.}
\makeatletter
\xpatchcmd\ST@read@cells
  {\ST@remove@first@spaces\ST@current@cell}
  {\ST@remove@first@spaces\ST@current@cell
  \StrDel\ST@current@cell,[\ST@current@cell]}{}{}
\makeatother
\let\STprintnum\numprint
\STautoround*{2} % Always show 2 decimals in spreadtab.

\begin{document}
\begin{tabular*}{\textwidth}{@ {}l @{\extracolsep{\fill}} r @{}}
  \begin{tabular}[b]{@{} l}
    {\fontsize{30}{30}\selectfont\textbf{\uppercase{Invoice}}}
  \end{tabular} &
  \begin{tabular}[b]{r @{}}
    \textbf{$issuer.name$} \\
    $for(issuer.address)$
    $issuer.address$ \\
    $endfor$
  \end{tabular}
\end{tabular*}

\vspace{1.8em}
\noindent\makebox[\linewidth]{\textcolor{LightGray}{{\rule{\paperwidth}{0.6pt}}}}%
\vspace{1.8em}

\noindent\textcolor{DimGray}{\large\textbf{\uppercase{Bill to}}} \\ \medskip
\begin{tabular*}{\textwidth}{@{} l @{\extracolsep{\fill}} r @{}}
  \begin{tabular}[t]{@{} l}
    \textbf{$recipient.name$} \\
    $for(recipient.address)$
    $recipient.address$ \\
    $endfor$
  \end{tabular} &
  \begin{tabular}[t]{@{} r @{\hspace{0.7em}} l @{}}
    \textbf{Invoice Number:} & $invoice-number$ \\
    \textbf{Invoice Date:} & $date$ \\
    \textbf{Payment Due:} & Upon receipt
  \end{tabular}
\end{tabular*}

\vspace{2em}

\renewcommand{\arraystretch}{1.8}
\newcounter{item}
  $if(convert-currency-to)$
    \begin{spreadtab}{{tabularx}{\textwidth}[t t t t]{c X r r}}
      @ \textbf{Item} & @ \textbf{Description} & @ \textbf{Price in $currency$} & @ \textbf{Price in $convert-currency-to$}\footnotemark \\ \hline
      $for(services)$
        @ \refstepcounter{item} \theitem &
        @ $services.description$
          $if(services.details)$ \newline \small $services.details$ $endif$ &
        $services.price$ &
        $services.price$ * $exchange-rate$ \\
      $endfor$
      \hline
      @ & @ & & \\[-2em]
      @ & @ & @ \multicolumn{1}{r}{\textbf{Amount Due ($convert-currency-to$):}} & \textbf{:={sum(d1:[0,-1])}}
    \end{spreadtab}
  $else$
    \begin{spreadtab}{{tabularx}{\textwidth}[t t t]{c X r}}
      @ \textbf{Item} & @ \textbf{Description} & @ \textbf{Price in $currency$} \\ \hline
      $for(services)$
        @ \refstepcounter{item} \theitem &
        @ $services.description$
          $if(services.details)$ \newline \small $services.details$ $endif$ &
        $services.price$ \\
      $endfor$
      \hline
      @ & @ & \\[-2em]
      @ & @ \multicolumn{1}{r}{\textbf{Amount Due ($currency$):}} & \textbf{:={sum(c1:[0,-1])}}
    \end{spreadtab}
  $endif$

$if(convert-currency-to)$
  \footnotetext{1 $currency$ = $exchange-rate$ $convert-currency-to$ as per www.xe.com on $date$.}
$endif$

\vspace{1.8em}
\noindent\makebox[\linewidth]{\textcolor{LightGray}{{\rule{\paperwidth}{0.6pt}}}}%
\vspace{1.8em}

\renewcommand{\arraystretch}{1.15}
\noindent\textcolor{DimGray}{\large\textbf{\uppercase{Bank details}}} \\ \medskip
\begin{tabular}[b]{@{} l l @{}}
  \textbf{Name:} & $issuer.bank-details.name$ \\
  \textbf{Address:} & $issuer.bank-details.address$ \\
  \textbf{Account Number:} & $issuer.bank-details.account-number$ \\
  \textbf{IBAN:} & $issuer.bank-details.iban$ \\
  \textbf{SWIFT / BIC:} & $issuer.bank-details.swift$
\end{tabular}

\end{document}
